\documentclass[12pt]{report}
\usepackage[spanish, activeacute]{babel}
\usepackage[top=2.75cm,bottom=2.50cm,left=3.00cm,right=2.50cm]{geometry}
\usepackage[utf8]{inputenc}  
\usepackage{enumerate}
\usepackage{graphicx}



\begin{document}
	\setlength{\topmargin}{-0.5in}
	\pagestyle{empty}
	\begin{center}
		\textbf{
			\vspace{-0.7em}
			ESCUELA SUPERIOR POLITÉCNICA DEL LITORAL
		}
		\line(1,0){380}\\		
		\scriptsize{FACULTAD DE INGENIERÍA EN ELECTRICIDAD Y COMPUTACIÓN}
	\end{center}
	\begin{center}
		\vspace{2.5em}
		Lenguajes de Programación
		\\2012 | II Término
		\vspace{1.5em}
		\\Ana Arias - acarias@espol.edu.ec
		\vspace{0.6mm}
		\\Liliana Ramos - ljramos@espol.edu.ec
		\\Denny Schuldt - dschuldt@espol.edu.ec
		\vspace{3em}
		\large{\textbf{\\ Observaciones, Conclusiones y Experiencias del Desarrollo	\vspace{2em}}}		
	\Huge{\textbf{\\ Comic It!	\vspace{1em}}}
\end{center}
		
\begingroup
		\large{
			\textbf{
				Objetivo General
				\newline
				\newline
			}
		}
	\endgroup

	Dar a conocer todas las experiencias vividas durante el desarrollo de nuestra aplicación para el sistema operativo Android.

	\vspace{4em}

\begingroup
		\large{
			\textbf{
				Observaciones
				\newline
				\newline
			}
		}
	\endgroup

		\begin{enumerate}[a]%for small alpha-characters within brackets.
		\item Android es un sistema operativo muy curioso. Puesto que está conformado por xmls y sus componentes se las programan en Java que extienden de una propiedad que se llama Activity.
Esto fue un gran reto para nosotros, puesto que no teníamos ni si quiera la noción de como esto trabajaba.
		\item Nuestro proyecto incluye muchas funcionalidades básicas del teléfono en el que se lo vaya a instalar. Usamos lo que es la cámara, la galería y arrastrar y soltar. En particular, esta última fue la que se nos hizo muy complicada, puesto que no solo queíamos que se arrastre una imagen, sino que también queríamos que se redimensione (zoom). Aún estamos trabajando en aquello para que quede perfecto.
		\end{enumerate}
	



	\vspace{4em}
	
	\begingroup
		\large{
			\textbf{
				Experiencias
				\newline
				\newline
			}
		}
	\endgroup

		\begin{enumerate}[--]%for small alpha-characters within brackets.
		\item Una de las experiencias o anécdotas que podemos contar, es acerca del manejo de los Layouts. Fue muy complicado trabajar con ellos. En especial cuando usabamos el drag and drop con el zoom. 
\newline Como ya deberíamos saber, cuando uno agrega algo a una ventana, este tiene su orden en el eje de las z. Cada una está encima de otra. Cuando trabajamos con estos Layouts opacaban que estaba atrás y no lo podíamos seleccionar. Fue muy cómico como nos fijamos en donde estaba el error, porque empezamos a probar varias cosas y ninguna nos resultó.
Por último, estamos pensando en ponerle las propiedades del drag a estos layouts y no a sus componentes, para que estos se muevan y dejen en libertad a los demás.
		\item .
		\end{enumerate}


	\vspace{4em}
	
	\begingroup
		\large{
			\textbf{
				Conclusiones
				\newline
				\newline
			}
		}
	\endgroup

		\begin{enumerate}[1]%for small alpha-characters within brackets.
		\item .
		\item .
		\end{enumerate}


	


\end{document}


